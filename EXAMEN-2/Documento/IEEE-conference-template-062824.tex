\documentclass[conference]{IEEEtran}
\IEEEoverridecommandlockouts
% The preceding line is only needed to identify funding in the first footnote. If that is unneeded, please comment it out.
%Template version as of 6/27/2024

\usepackage{cite}
\usepackage{amsmath,amssymb,amsfonts}
\usepackage{algorithmic}
\usepackage{graphicx}
\usepackage{svg}
\usepackage{textcomp}
\usepackage{xcolor}
\usepackage{amsmath}
\def\BibTeX{{\rm B\kern-.05em{\sc i\kern-.025em b}\kern-.08em
    T\kern-.1667em\lower.7ex\hbox{E}\kern-.125emX}}
\begin{document}

\title{Examen 2: Sistemas Avanzados de medici\'on\\
	\thanks{Identify applicable funding agency here. If none, delete this.}
}

\author{\IEEEauthorblockN{1\textsuperscript{st} Juan Camil Vasco Leiva}
	\IEEEauthorblockA{\textit{Maestr\'ia en Ingenier\'ia El\'ectrica} \\
		\textit{Universidad T\'ecnologia de Pereira}\\
		Pereira, Colombia \\
		camilo.vasco@utp.edu.co}
	\and
	\IEEEauthorblockN{2\textsuperscript{nd} Juan Diego G\'omez Chavarro}
	\IEEEauthorblockA{\textit{Maestr\'ia en Ingenier\'ia El\'ectrica} \\
		\textit{Universidad T\'ecnologica de Pereira}\\
		Periera, Colombia \\
		juandiego.gomez1@utp.edu.co}

}

\maketitle

\begin{IEEEkeywords}
	Filtro de kalman,
\end{IEEEkeywords}

\section{Introducción}

En este trabajo se realizar\'a la recuperacion de la se\~nal de un sensor de segundo orden, el cual es un sensor de fuerza ...

\section{Desarrollo}


\textbf{ENUNCIADO:}



\subsection{Modelo}

% Modelado en Espacio de Estados
% Sensor de Segundo Orden
\begin{equation}
	\frac{d^2y}{dt^2} + 2\zeta\omega_n \frac{dy}{dt} + \omega_n^2 y = \frac{K_s}{m} F(t)
\end{equation}


Donde la se\~nal $y(t)$ es la salida del sensor, $\zeta$ es un coeficiente de amortiguamiento, $\omega_n$ se conoce como la frecuencia natual de oscilaci\'on del sistema, ${K_s}$ es una constante de rigidez y $m$ es una masa asociada al sistema de amortiguamiento.

Definiendo las variables de estado:
\[
	x_1 = y
\]
\[
	x_2 = \frac{dy}{dt}
\]

Se obtiene el modelo en espacio de estados:

\[
	\dot{x}_1 = x_2
\]
\[
	\dot{x}_2 = -2\zeta\omega_n x_2 - \omega_n^2x_1 + \frac{K_s}{m}F(t)
\]


En forma matricial:

\begin{equation*}
	\begin{bmatrix}
		\dot{x}_1 \\
		\dot{x}_2
	\end{bmatrix}
	=
	\begin{bmatrix}
		0           & 1               \\
		-\omega_n^2 & -2\zeta\omega_n
	\end{bmatrix}
	\begin{bmatrix}
		x_1 \\
		x_2
	\end{bmatrix}
	+
	\begin{bmatrix}
		0 \\
		\frac{K_s}{m}
	\end{bmatrix}
	F(t)
\end{equation*}


A partir del modelo dinamico es posible recuperar $F(t)$, as\'i:
\begin{equation*}
	\textbf{F}(t) = \frac{m}{K_s} \left( \dot{x}_2 + 2\zeta\omega_n x_2 + \omega_n^2 x_1 \right)
\end{equation*}


Donde $\dot{x}_2$ se aproxima mediante el m\'etodo de euler.

\begin{equation*}
	\dot{x}_2 \approx \frac{x_2[n] - x_2[n-1]}{\Delta t}
\end{equation*}

\begin{equation*}
	\textbf{A} =
	\begin{bmatrix}
		0           & 1               \\
		-\omega_n^2 & -2\zeta\omega_n
	\end{bmatrix}
	, \textbf{B} =
	\begin{bmatrix}
		0 \\
		\frac{K_s}{m}
	\end{bmatrix}
	, \textbf{C} =
	\begin{bmatrix}
		1 & 0
	\end{bmatrix}
	, \textbf{D} = 0
\end{equation*}


\subsection{Hardware in the loop}

Imagen generada mediante hardaware in the loop(Borra esta fras)
\begin{figure}[h]
    \centering
    \includegraphics[width=0.5\textwidth]{../CodigoPython/imagenes/SegundoOrden.png}
    \caption{Descripción de la imagen}
    \label{fig:etiqueta}
\end{figure}

\subsection{Filtro de Kalman Lineal(LKF)}

El filtro de Kalman estima los estados $\textbf{x} = [x_1, x_2]^T$y la entrada $F(t)$, asumiendo un modelo lineal:

\[
	\mathbf{x}[n + 1] = \textbf{F} \textbf{x}[n] + \textbf{B}\textbf{F}[n] + \textbf{w} [n]
\]

\[
	\textbf{z}[n] = \textbf{H} \textbf{x}[n] + \textbf{v}[n]
\]

Donde $w$ y $v$ son ruido del proceso y medici\'on repectivamente. El LKF minimiza el error cuadr\'atico medio mediante dos etapas.

\subsubsection{Predicci\'on}


\[
	\hat{\mathbf{x}}_k^- = \mathbf{F}\hat{\mathbf{x}^+_{k-1}} + \mathbf{B}u_{k-1}
\]
\[
	P^-_k = F P_{k-1}F^T + Q
\]

\subsubsection{Actualizaci\'on}

\[
	\mathbf{K}_k = \mathbf{P}_k^- \mathbf{H}^T \left( \mathbf{H} \mathbf{P}_k^- \mathbf{H}^T + \mathbf{R} \right)^{-1}
\]

\[
	\hat{\mathbf{x}}_k^+ = \hat{\mathbf{x}}_k^- + \mathbf{K}_k \left( \mathbf{z}_k - \mathbf{H} \hat{\mathbf{x}}_k^- \right)
\]

\[
	\mathbf{P}_k^+ = \left( \mathbf{I} - \mathbf{K}_k \mathbf{H} \right) \mathbf{P}_k^-
\]

\subsection{Estimaciones}

\subsubsection{Estaimacio\'on $x_1$}

\begin{figure}[h]
	\centering
	\includegraphics[width=0.5\textwidth]{../CodigoPython/imagenes/EstimacionX1.png}
	\caption{Descripción de la imagen.}
	\label{fig:etiqueta}
\end{figure}

\subsubsection{Estimaci\'on $x_2$}
\begin{figure}[h]
	\centering
	\includegraphics[width=0.5\textwidth]{../CodigoPython/imagenes/EstimacionX2.png}
	\caption{Descripción de la imagen.}
	\label{fig:etiqueta}
\end{figure}

\subsubsection{Estimaci\'on $F(t)$}
\begin{figure}[h]
	\centering
	\includegraphics[width=0.5\textwidth]{../CodigoPython/imagenes/EstimacionF.png}
	\caption{Descripción de la imagen.}
	\label{fig:etiqueta}
\end{figure}

\section{Conclusiones}

Aqui van las conclusiones

\begin{thebibliography}{00}

	\item Clases magistrales e información suministrada en la clase de Sistemas de medición avanzada, Maestria en ingenieria electrica, semestre 2025-1

\end{thebibliography}

\end{document}
