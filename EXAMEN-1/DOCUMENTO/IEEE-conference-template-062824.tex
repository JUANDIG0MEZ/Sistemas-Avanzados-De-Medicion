\documentclass[conference]{IEEEtran}
\IEEEoverridecommandlockouts
% The preceding line is only needed to identify funding in the first footnote. If that is unneeded, please comment it out.
%Template version as of 6/27/2024

\usepackage{cite}
\usepackage{amsmath,amssymb,amsfonts}
\usepackage{algorithmic}
\usepackage{graphicx}
\usepackage{svg}
\usepackage{textcomp}
\usepackage{xcolor}
\usepackage{amsmath}
\def\BibTeX{{\rm B\kern-.05em{\sc i\kern-.025em b}\kern-.08em
    T\kern-.1667em\lower.7ex\hbox{E}\kern-.125emX}}
\begin{document}

\title{Examen 1: Sistemas Avanzados de medici\'on\\
\thanks{Identify applicable funding agency here. If none, delete this.}
}

\author{\IEEEauthorblockN{1\textsuperscript{st} Juan Camilo Vasco Leiva}
\IEEEauthorblockA{\textit{Maestr\'ia en Ingenier\'ia El\'ectrica} \\
\textit{Universidad T\'ecnologia de Pereira}\\
Pereira, Colombia \\
camilo.vasco@utp.edu.co}
\and
\IEEEauthorblockN{2\textsuperscript{nd} Juan Diego G\'omez Chavarro}
\IEEEauthorblockA{\textit{Maestr\'ia en Ingenier\'ia El\'ectrica} \\
\textit{Universidad T\'ecnologica de Pereira}\\
Periera, Colombia \\
juandiego.gomez1@utp.edu.co}

}

\maketitle

\begin{abstract}
This document is a model and instructions for \LaTeX.
This and the IEEEtran.cls file define the components of your paper [title, text, heads, etc.]. *CRITICAL: Do Not Use Symbols, Special Characters, Footnotes, 
or Math in Paper Title or Abstract.
\end{abstract}

\begin{IEEEkeywords}
Ajuste, error, rango, sensores
\end{IEEEkeywords}

\section{Introducción}

En este examen práctico de sistemas avanzados de medición, se realizará el ajuste fino de curvas de sensores dentro de un rango especificado por el usuario. El objetivo es simular una ejecución real de una curva de temperatura durante el proceso de horneado de un elemento "X". Para este ejercicio, se considerarán diversas características, como los diferentes errores inherentes a la medición de cada sensor, incluyendo errores de calibración, errores homocedásticos y heterocedásticos, así como la presencia de valores atípicos (outliers) en las mediciones.



\section{Desarrollo}

A continuación se desarrollaran cada uno de los pasos para la ejecución del examen realizado en python para el desarrollo de los puntos dictados para el ejercicio:\\

\textbf{ENUNCIADO:}\\

En un horno industrial, hay 4 sensores de temperatura diferentes instalados, cuyos datasheets son conocidos y contienen los datos del fabricante para determinar su curva característica. \\

Seleccione los cuatro sensores de su preferencia, con base en los datasheets adjuntos. Determine la curva característica de cada sensor; luego, simule el proceso de adquisición de datos para los cuatro sensores, asumiendo que cada sensor presenta ruido que no necesariamente es gaussiano, así:\\

\begin{itemize}
	\item Seleccione un rango de operación de cada sensor que no supere el 60% del rango total del dispositivo. 
	\item Encuentre la ecuación característica exclusivamente en dicho rango y determine el intervalo de incertidumbre al usar el ajuste de su ecuación.
	\item Simule que el horno sigue algún perfil de temperatura conocido. Por ejemplo, la temperatura incrementa X grados en Y segundos, para luego decrecer Z grados en W segundos. 
	\item Simule la operación de cada sensor asumiendo primero comportamiento gaussiano y luego algún otro comportamiento (diferente en cada sensor).
	\item Su simulador debe tener la capacidad de simular outliers a la salida del sistema de medición. 
\end{itemize}


\subsection{Determinación de curva caracteristica}

Lo primero que se debe realizar, es una extracción de los datos dados por los diversos fabricantes en los datasheets del sensor, y que se pueden encontrar en las tablas de cada datasheet, dichas tablas expresan el valor de salida del sensor según la temperatura de que esta midiendo. Para esto hemos creado un diccionario, con el nombre de cada sensor y sus respectivos valores de Temperatura y valor de medida (mV, V, $\omega$, ..., etc).\\

Una vez se tiene el diccionario correspondiente, se procede a graficar cada uno de los cuatro (4) sensores, con lo cual es posible de manera visual-analitica, determinar el comportamiento de medida de cada sensor:


\begin{figure}[h!]
	\centering
	\includegraphics[width=0.8\columnwidth]{PT1000_con_ajuste.png}
	\caption{Curva rango completo PT1000}
	\label{fig:1}
\end{figure}

\begin{figure}[h!]
	\centering
	\includegraphics[width=0.8\columnwidth]{TYPE_E_con_ajuste.png}
	\caption{Curva rango completo TYPE E}
	\label{fig:2}
\end{figure}

\begin{figure}[h!]
	\centering
	\includegraphics[width=0.8\columnwidth]{TMP235_con_ajuste.png}
	\caption{Curva rango completo TYPE TMP}
	\label{fig:3}
\end{figure}

\begin{figure}[h!]
	\centering
	\includegraphics[width=0.8\columnwidth]{NTCLE100E3_con_ajuste.png}
	\caption{Curva rango completo NTCLE100E3338}
	\label{fig:4}
\end{figure}

De lo cual es posible determinar que:\\

\begin{itemize}
	\item El sensor PT1000 es de curva lineal
	\item El sensor TYPE E es de curva polinomial
	\item El sensor TYPE TMP es de curva lineal
	\item El sensor NTCLE100E3338 es de curva exponencial
\end{itemize}



\subsection{Determinación de curva caracteristica}

Para la determinación del rango de operación, y que este dentro de por lo menos el 60\% de todos los sensores, es necesario graficar todos los rangos de los sensores y traslaparlos, con el fin de encontrar un rango que todos posean dentro de sus datasheets, y que tambien sean de maximo el 60\%, a continuación la grafica y el rango escogido para la curva de temperatura del horno a experimentar con el elementos "X",

\begin{figure}[h!]
	\centering
	\includegraphics[width=0.8\columnwidth]{rangos_sensores.png}
	\caption{Rangos de temperatura por sensor}
	\label{fig:5}
\end{figure}

El rango definido (rango deseado en la grafica \ref{fig:5}) para la simulación del horno y para tener una igualdad de rangos en el margen de los sensores según los rangos dados es de 0° a 100° Centigrados. Lo cual en una aplicación practica puede estar relacionado con un rango de cocina de un alimento, como por ejemplo la pasteurización de la leche (donde la leche se lleva a una temperatura que oscila entre los 55 y los 75 ºC durante 17 segundos).\\

Es posible analizar las magmnitudes de medición del rango de cada sensor, y que el rango escogido bajo las especificaciones del ejercicio, subdimensiona o le da bajo rendimiento a los sensores de mayor rango.


\subsection{Ecuación caracteristica para rango especifico}

En este apartado, es necesario aplicar un ajuste fino a cada uno de nuestros sensores, esto debido a que ya tenemos dos datos de gran importancia, el primero saber la curva de comportamiento de cada sensor y segundo el rango en el que vamos a trabajar, esta es una practica de uso general en aplicaciones de medición en todas las escalas, debido a que el fabricante propone una tabla de uso general y en el mayor rango posible, sin embargo, las aplicaciones dadas por los diferentes usuarios necesitan un ajuste o una curva de gran presicion en el proceso especifico.\\

Para este ejercicio y los sensores escogidos las curvas generales fueron:\\

$lineal: R(T) = m*T + b ==> [T, 1] * [[m], [b]]$\\

$exponencial: R(T) = A * e^(B/T) ==> ln(R(T)) = ln(A) + B/T ==> ln(R(T)) = [1/T, 1] * [[B], [ln(A)]]$\\
"""

$polinomial: V(T) = A + B * T + C * T^2 + D * T^3$\\

Los parametros obtenido para cada uno de los sensores en el rango fueron:


\begin{table}[h]
	\centering
	\caption{Párametros de rango ajustado de cada sensor}
	\label{tab:my-table1}
	\resizebox{\columnwidth}{!}{%
		\begin{tabular}{|c|c|c|c|c|c|}
			\hline
			\textbf{Sensor}              & \textbf{Curva} & \textbf{A}         & \textbf{B}         & \textbf{C}         & \textbf{D}          \\ \hline
			PT1000                       & Lineal         & 3.85               & 1000.92            & -                  & -                   \\ \hline
			\multicolumn{1}{|c|}{TYPE E} & Polinomial     & $4.89\cdot e^{-5}$ & $5.86\cdot e^{-2}$ & $4.74\cdot e^{-5}$ & $-1.85\cdot e^{-8}$ \\ \hline
			TYPE TMP                     & Lineal         & 10                 & 500                & -                  & -                   \\ \hline
			NTCLE100E3338                & Exponencial    & 2873.3             & -8.45              & -                  & -                   \\ \hline
		\end{tabular}%
	}
\end{table}

% Please add the following required packages to your document preamble:
% \usepackage{graphicx}
\begin{table}[h]
	\centering
	\caption{Error MSE (mean square error) por sensor}
	\label{tab:my-table2}
	\resizebox{0.5\columnwidth}{!}{%
		\begin{tabular}{|c|c|}
			\hline
			\textbf{Sensor} & \textbf{Error MSE} \\ \hline
			PT1000          & 0.4707             \\ \hline
			TYPE E          & 0.0002             \\ \hline
			TYPE TMP        & 1.6126             \\ \hline
			NTCLE100E3338   & 0.3561             \\ \hline
		\end{tabular}%
	}
\end{table}

\subsection{Propagaci\'on de incertidumbre}

Dado que algunos de los fabricantes presentan el error en t\'erminos de temperatura, es necesario conocer el impacto en la medida del sensor, ya sea en t\'erminos de voltaje o resistencia. Por ende, se utilizar\'a la fo\'rmula de propagac\'on de incertidumbre de primero orden de una variable.

\begin{equation}
	\sigma_f^2 = \left( \frac{df}{dT} \right)^2 \sigma_T^2
\end{equation}

donde:
\begin{itemize}
	\item $\sigma_f^2$ es la varianza de la funci\'on $f$,
	\item $\frac{d f}{d x}$ es la derivada de la funci\'on respecto a la variable $x$,
	\item $\sigma_x^2$ es la varianza de la variable $x$.
\end{itemize}



\subsubsection{Funci\'on lineal}
	\[
		f(T) = A * T + B
	\]
	\[
		\frac{d f}{d x} = A
	\]
	\[
		\sigma_f^2 = A^2 \sigma_T^2
	\]

\subsubsection{Funci\'on exponencial}
	\[
	f(T) = A \cdot e^{\left( \frac{B}{T} \right)}
	\]
	\[
	\frac{d f}{d T} = A \cdot e^{\left( \frac{B}{T} \right)} \cdot \left( -\frac{B}{T^2} \right)
	\]
	\[
	\frac{d f}{d T} = -\frac{A B}{T^2} \cdot e^{\left( \frac{B}{T} \right)}
	\]
	\[
	\sigma_f^2 = \left( \frac{d f}{d T} \right)^2 \sigma_T^2
	\]
	\[
	\sigma_f^2 = \left( -\frac{A B}{T^2} e^{\left( \frac{B}{T} \right)} \right)^2 \sigma_T^2
	\]
	\[
	\sigma_f^2 = A^2 B^2 \frac{e^{\left( \frac{2B}{T} \right)}}{T^4} \sigma_T^2
	\]

\subsubsection{Funci\'on polinomial}
	
	\[
	f(T) = A + B \cdot T + C \cdot T^2 + D \cdot T^3
	\]
	Calculamos la derivada de \( F(T) \) respecto a \( T \):
	\[
	\frac{d f}{d T} = B + 2 C T + 3 D T^2
	\]
	Luego, aplicando la fórmula de propagación de incertidumbre:
	\[
	\sigma_f^2 = \left( \frac{d f}{d T} \right)^2 \sigma_T^2
	\]
	por lo tanto:
	\[
	\sigma_f^2 = \left( B + 2 C T + 3 D T^2 \right)^2 \sigma_T^2
	\]
	

\subsection{Errores}
	Cada fabricante nos presenta el error propio de cada sensor, este depende del tipo de medición, materiales tecnologia de construcción y demás, y luego probados en laboratorio, alli mediante un sensor de referencia de la más alta calidad y precisión es posible obtener los errores homocedasticos y heterocedasticos de cada sensor de la siguiente tabla.
	
	\begin{table}[h]
		\centering
		\begin{tabular}{|c|c|}
			\hline
			Sensor & Error \\ \hline
			PT1000 & $\pm 0.5\ ^{\circ} \mathrm{C}$ \\ \hline
			Tipo K & 
			\begin{tabular}[c]{@{}c@{}}
				Si $T > 0\ ^{\circ}\mathrm{C}$: $\pm 2.2\ ^{\circ}\mathrm{C}$ o $\pm 0.75\%$, el que sea mayor. \\ 
				Si $T < 0\ ^{\circ}\mathrm{C}$: $\pm 2.2\ ^{\circ}\mathrm{C}$ o $\pm 2.0\%$, el que sea mayor.
			\end{tabular} \\ \hline
			Tipo E & 
			\begin{tabular}[c]{@{}c@{}}
				Si $T > 0\ ^{\circ}\mathrm{C}$: $\pm 1.7\ ^{\circ}\mathrm{C}$ o $\pm 0.5\%$, el que sea mayor. \\ 
				Si $T < 0\ ^{\circ}\mathrm{C}$: $\pm 1.7\ ^{\circ}\mathrm{C}$ o $\pm 1.0\%$, el que sea mayor.
			\end{tabular} \\ \hline
			NTCLE100E3 & $ \pm 2\%$ sobre la medici\'on. \\ \hline
			
		\end{tabular} 
			
		\caption{Errores proporcionados por los fabricantes para cada sensor}
		\label{tab:errores_sensores}
	\end{table}


\subsection{Error de medida}
	En nuestro caso, las medidas de cada sensor se ven afectadas por dos factores, tanto el error del ajuste como el error del fabricante. Dicho lo anterior, para encontrar el error se realizara una suma de errores.
	
	\[
		\Delta f = \pm \sqrt{\Delta f_{fabricante} + \Delta f_{ajuste}}
	\]
	
	
	
	
	
\subsection{Simulación de horno}

En este punto es necesario entonces que mediante el rango escogido para los sensores se realice una curva caracteristica del proceso, en este caso la curva es:

\textbf{INSERTAR CURVA ESCOGIDA}

Es en esta curva donde se probara el comportamiento y la salida de cada uno de los sensores

\subsection{Operación de los sensores}

Para la operación de los sensores se asumira un comportamiento Gaussiano para todos, y luego un comportamiento de medida diferente para cada uno de los sensores, en este proceso es posible la inclusión de outliers con el 0.5\%

\subsubsection{Comportamiento Gaussiano}

\subsubsection{Comportamiento Alterno}


\section{Conclusiones}


\begin{thebibliography}{00}

\item Clases magistrales e información suministrada en la clase de Sistemas de medición avanzada, Maestria en ingenieria electrica, semestre 2025-1

\end{thebibliography}

\end{document}
