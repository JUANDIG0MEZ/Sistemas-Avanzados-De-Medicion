\documentclass[conference]{IEEEtran}
\IEEEoverridecommandlockouts
% The preceding line is only needed to identify funding in the first footnote. If that is unneeded, please comment it out.
%Template version as of 6/27/2024

\usepackage{cite}
\usepackage{amsmath,amssymb,amsfonts}
\usepackage{algorithmic}
\usepackage{graphicx}
\usepackage{textcomp}
\usepackage{xcolor}
\def\BibTeX{{\rm B\kern-.05em{\sc i\kern-.025em b}\kern-.08em
    T\kern-.1667em\lower.7ex\hbox{E}\kern-.125emX}}
\begin{document}

\title{Examen 1: Sistemas Avanzados de medici\'on\\
\thanks{Identify applicable funding agency here. If none, delete this.}
}

\author{\IEEEauthorblockN{1\textsuperscript{st} Juan Camilo Vasco Leiva}
\IEEEauthorblockA{\textit{Maestr\'ia en Ingenier\'ia El\'ectrica} \\
\textit{Universidad T\'ecnologia de Pereira}\\
Pereira, Colombia \\
camilo.vasco@utp.edu.co}
\and
\IEEEauthorblockN{2\textsuperscript{nd} Juan Diego G\'omez Chavarro}
\IEEEauthorblockA{\textit{Maestr\'ia en Ingenier\'ia El\'ectrica} \\
\textit{Universidad T\'ecnologica de Pereira}\\
Periera, Colombia \\
juandiego.gomez1@utp.edu.co}

}

\maketitle

\begin{abstract}
This document is a model and instructions for \LaTeX.
This and the IEEEtran.cls file define the components of your paper [title, text, heads, etc.]. *CRITICAL: Do Not Use Symbols, Special Characters, Footnotes, 
or Math in Paper Title or Abstract.
\end{abstract}

\begin{IEEEkeywords}
Ajuste, error, rango, sensores
\end{IEEEkeywords}

\section{Introducción}

En este examen práctico de sistemas avanzados de medición, se realizará el ajuste fino de curvas de sensores dentro de un rango especificado por el usuario. El objetivo es simular una ejecución real de una curva de temperatura durante el proceso de horneado de un elemento "X". Para este ejercicio, se considerarán diversas características, como los diferentes errores inherentes a la medición de cada sensor, incluyendo errores de calibración, errores homocedásticos y heterocedásticos, así como la presencia de valores atípicos (outliers) en las mediciones.



\section{Desarrollo}

A continuación se desarrollaran cada uno de los pasos para la ejecución del examen realizado en python para el desarrollo de los puntos dictados para el ejercicio:\\

\textbf{ENUNCIADO:}\\

En un horno industrial, hay 4 sensores de temperatura diferentes instalados, cuyos datasheets son conocidos y contienen los datos del fabricante para determinar su curva característica. \\

Seleccione los cuatro sensores de su preferencia, con base en los datasheets adjuntos. Determine la curva característica de cada sensor; luego, simule el proceso de adquisición de datos para los cuatro sensores, asumiendo que cada sensor presenta ruido que no necesariamente es gaussiano, así:\\

\begin{itemize}
	\item Seleccione un rango de operación de cada sensor que no supere el 60% del rango total del dispositivo. 
	\item Encuentre la ecuación característica exclusivamente en dicho rango y determine el intervalo de incertidumbre al usar el ajuste de su ecuación.
	\item Simule que el horno sigue algún perfil de temperatura conocido. Por ejemplo, la temperatura incrementa X grados en Y segundos, para luego decrecer Z grados en W segundos. 
	\item Simule la operación de cada sensor asumiendo primero comportamiento gaussiano y luego algún otro comportamiento (diferente en cada sensor).
	\item Su simulador debe tener la capacidad de simular outliers a la salida del sistema de medición. 
	\item 
\end{itemize}


\subsection{Determinación de curva caracteristica}

Lo primero que se debe realizar, es una extracción de los datos dados por los diversos fabricantes en los datasheets del sensor, y que se pueden encontrar en las tablas de cada datasheet, dichas tablas expresan el valor de salida del sensor según la temperatura de que esta midiendo. Para esto hemos creado un diccionario, con el nombre de cada sensor y sus respectivos valores de Temperatura y valor de medida (mV, V, $\omega$, ..., etc)

Una vez se tiene el diccionario correspondiente, se procede a graficar cada uno de los cuatro (4) sensores, con lo cual es posible de manera visual-analitica, determinar el comportamiento de medida de cada sensor:

\textbf{INSERTAR IMAGENES EN REFERENCIA}

De lo cual es posible determinar que:

XXX es XXX

\subsection{Determinación de curva caracteristica}

Para la determinación del rango de operación





\section*{References}

Please number citations consecutively within brackets \cite{b1}. The 
sentence punctuation follows the bracket \cite{b2}. Refer simply to the reference 
number, as in \cite{b3}---do not use ``Ref. \cite{b3}'' or ``reference \cite{b3}'' except at 
the beginning of a sentence: ``Reference \cite{b3} was the first $\ldots$''

Number footnotes separately in superscripts. Place the actual footnote at 
the bottom of the column in which it was cited. Do not put footnotes in the 
abstract or reference list. Use letters for table footnotes.

Unless there are six authors or more give all authors' names; do not use 
``et al.''. Papers that have not been published, even if they have been 
submitted for publication, should be cited as ``unpublished'' \cite{b4}. Papers 
that have been accepted for publication should be cited as ``in press'' \cite{b5}. 
Capitalize only the first word in a paper title, except for proper nouns and 
element symbols.

For papers published in translation journals, please give the English 
citation first, followed by the original foreign-language citation \cite{b6}.

\begin{thebibliography}{00}
\bibitem{b1} G. Eason, B. Noble, and I. N. Sneddon, ``On certain integrals of Lipschitz-Hankel type involving products of Bessel functions,'' Phil. Trans. Roy. Soc. London, vol. A247, pp. 529--551, April 1955.
\bibitem{b2} J. Clerk Maxwell, A Treatise on Electricity and Magnetism, 3rd ed., vol. 2. Oxford: Clarendon, 1892, pp.68--73.
\bibitem{b3} I. S. Jacobs and C. P. Bean, ``Fine particles, thin films and exchange anisotropy,'' in Magnetism, vol. III, G. T. Rado and H. Suhl, Eds. New York: Academic, 1963, pp. 271--350.
\bibitem{b4} K. Elissa, ``Title of paper if known,'' unpublished.
\bibitem{b5} R. Nicole, ``Title of paper with only first word capitalized,'' J. Name Stand. Abbrev., in press.
\bibitem{b6} Y. Yorozu, M. Hirano, K. Oka, and Y. Tagawa, ``Electron spectroscopy studies on magneto-optical media and plastic substrate interface,'' IEEE Transl. J. Magn. Japan, vol. 2, pp. 740--741, August 1987 [Digests 9th Annual Conf. Magnetics Japan, p. 301, 1982].
\bibitem{b7} M. Young, The Technical Writer's Handbook. Mill Valley, CA: University Science, 1989.
\bibitem{b8} D. P. Kingma and M. Welling, ``Auto-encoding variational Bayes,'' 2013, arXiv:1312.6114. [Online]. Available: https://arxiv.org/abs/1312.6114
\bibitem{b9} S. Liu, ``Wi-Fi Energy Detection Testbed (12MTC),'' 2023, gitHub repository. [Online]. Available: https://github.com/liustone99/Wi-Fi-Energy-Detection-Testbed-12MTC
\bibitem{b10} ``Treatment episode data set: discharges (TEDS-D): concatenated, 2006 to 2009.'' U.S. Department of Health and Human Services, Substance Abuse and Mental Health Services Administration, Office of Applied Studies, August, 2013, DOI:10.3886/ICPSR30122.v2
\bibitem{b11} K. Eves and J. Valasek, ``Adaptive control for singularly perturbed systems examples,'' Code Ocean, Aug. 2023. [Online]. Available: https://codeocean.com/capsule/4989235/tree
\end{thebibliography}

\vspace{12pt}
\color{red}
IEEE conference templates contain guidance text for composing and formatting conference papers. Please ensure that all template text is removed from your conference paper prior to submission to the conference. Failure to remove the template text from your paper may result in your paper not being published.

\end{document}
