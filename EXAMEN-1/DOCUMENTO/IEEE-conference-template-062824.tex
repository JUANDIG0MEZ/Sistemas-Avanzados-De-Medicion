\documentclass[conference]{IEEEtran}
\IEEEoverridecommandlockouts
% The preceding line is only needed to identify funding in the first footnote. If that is unneeded, please comment it out.
%Template version as of 6/27/2024

\usepackage{cite}
\usepackage{amsmath,amssymb,amsfonts}
\usepackage{algorithmic}
\usepackage{graphicx}
\usepackage{svg}
\usepackage{textcomp}
\usepackage{xcolor}
\usepackage{amsmath}
\def\BibTeX{{\rm B\kern-.05em{\sc i\kern-.025em b}\kern-.08em
    T\kern-.1667em\lower.7ex\hbox{E}\kern-.125emX}}
\begin{document}

\title{Examen 1: Sistemas Avanzados de medici\'on\\
\thanks{Identify applicable funding agency here. If none, delete this.}
}

\author{\IEEEauthorblockN{1\textsuperscript{st} Juan Camilo Vasco Leiva}
\IEEEauthorblockA{\textit{Maestr\'ia en Ingenier\'ia El\'ectrica} \\
\textit{Universidad T\'ecnologia de Pereira}\\
Pereira, Colombia \\
camilo.vasco@utp.edu.co}
\and
\IEEEauthorblockN{2\textsuperscript{nd} Juan Diego G\'omez Chavarro}
\IEEEauthorblockA{\textit{Maestr\'ia en Ingenier\'ia El\'ectrica} \\
\textit{Universidad T\'ecnologica de Pereira}\\
Periera, Colombia \\
juandiego.gomez1@utp.edu.co}

}

\maketitle

\begin{IEEEkeywords}
Ajuste, error, rango, sensores
\end{IEEEkeywords}

\section{Introducción}

En este examen práctico de sistemas avanzados de medición, se realizará el ajuste fino de curvas de sensores dentro de un rango especificado por el usuario. El objetivo es simular una ejecución real de una curva de temperatura durante el proceso de horneado de un elemento "X". Para este ejercicio, se considerarán diversas características, como los diferentes errores inherentes a la medición de cada sensor, incluyendo errores de calibración, errores homocedásticos y heterocedásticos, así como la presencia de valores atípicos (outliers) en las mediciones.



\section{Desarrollo}

A continuación se desarrollaran cada uno de los pasos para la ejecución del examen realizado en python para el desarrollo de los puntos dictados para el ejercicio:\\

\textbf{ENUNCIADO:}\\

En un horno industrial, hay 4 sensores de temperatura diferentes instalados, cuyos datasheets son conocidos y contienen los datos del fabricante para determinar su curva característica. \\

Seleccione los cuatro sensores de su preferencia, con base en los datasheets adjuntos. Determine la curva característica de cada sensor; luego, simule el proceso de adquisición de datos para los cuatro sensores, asumiendo que cada sensor presenta ruido que no necesariamente es gaussiano, así:\\

\begin{itemize}
	\item Seleccione un rango de operación de cada sensor que no supere el 60% del rango total del dispositivo. 
	\item Encuentre la ecuación característica exclusivamente en dicho rango y determine el intervalo de incertidumbre al usar el ajuste de su ecuación.
	\item Simule que el horno sigue algún perfil de temperatura conocido. Por ejemplo, la temperatura incrementa X grados en Y segundos, para luego decrecer Z grados en W segundos. 
	\item Simule la operación de cada sensor asumiendo primero comportamiento gaussiano y luego algún otro comportamiento (diferente en cada sensor).
	\item Su simulador debe tener la capacidad de simular outliers a la salida del sistema de medición. 
\end{itemize}




\subsection{Extracci\'on de datos}

Lo primero que se debe realizar, es una extracción de los datos dados por los diversos fabricantes en los datasheets del sensor, y que se pueden encontrar en las tablas de cada datasheet, dichas tablas expresan el valor de salida del sensor según la temperatura de que esta midiendo. Para esto hemos creado un diccionario, con el nombre de cada sensor y sus respectivos valores de Temperatura y valor de medida (mV, V, $\omega$, ..., etc).\\

\subsection{Curvas caracter\'isticas}

Para determinar la curva caracter\'istica que modela el comportamiento de los sensores en funci\'on de la temperatura, se debe consultar el datasheet o analizar el comportamiento del sensor mediante la gr\'afica de los datos extraidos. A partir de ello, se puede establecer lo siguiente (ver Tabla \ref{tab:tipoCurva}).

\begin{table}[h]
	\centering
	\caption{Tipo de curva}
	\label{tab:tipoCurva}
	\resizebox{0.5\columnwidth}{!}{%
		\begin{tabular}{|c|c|}
			\hline
			\textbf{Sensor} & \textbf{Tipo curva} \\ \hline
			PT1000          & Lineal            \\ \hline
			TYPE E          & Polinomial             \\ \hline
			TYPE TMP        & Lineal            \\ \hline
			NTCLE100E3338   & Exponencial            \\ \hline
		\end{tabular}%
	}
	
\end{table}

Cada tipo de curva sera expresada de la siguiente manera.

\subsubsection{Lineal}

\[
R(T) = A \cdot T + B
\]

\subsubsection{Exponencial}
\[
R(T) = A \cdot e^{(B/T)}
\]

\subsubsection{Polinomial}
\[
V(T) = A + B * T + C * T^2 + D * T^3
\]

\subsection{Descomposici\'on de valores singulares}

El m\'etodo SVD permite descomponer cualquier matriz \textbf{\textit{A}} de la forma:


\[
	svd(\mathbf{A}) = \mathbf{U} \mathbf{D} \mathbf{V}^T
\]

SVD es ampliamente utilizado para encontrar la solución de sistemas lineales sobredeterminados \( \mathbf{A}\mathbf{x} = \mathbf{b} \), minimizando el error cuadrático.
Una vez factorizada la matriz $\mathbf{A}$ del sistema lineal, es posible encontrar la soluci\'on de la siguiente forma:
\[
\mathbf{x} = \mathbf{A}^+ \mathbf{b}
\]


Donde $\mathbf{A}^{+} = \mathbf{V}\mathbf{D}^+ \mathbf{U}^T$ y $\mathbf{D}^+$ es 

\[
D_i^+ = 
\begin{cases}
	0 & \text{si } d_i = 0 \\
	\frac{1}{d_i} & \text{si } d_i \ne 0
\end{cases}
\]

Para utilizar el m\'etodo SVD y encontrar los coeficientes de las curvas caracteristicas, es necesario expresar estos modelos como un sistema lineal.


\subsubsection{Lineal}

\[
	R(T) = A \cdot T + B
\]

\[
	\begin{bmatrix} 
		T_{i} & 1 
	\end{bmatrix}
	\begin{bmatrix} 
		A \\ 
		B
	\end{bmatrix}
	= 
	\begin{bmatrix} 
		R(T_{i})
	\end{bmatrix}
\]


\subsubsection{Exponencial}
\[
	R(T) = A \cdot e^{(B/T)}
\]

\[
	ln(R(T)) = ln(A) + B/T
\]

\[
	\begin{bmatrix} 
		1/T_i & 1
	\end{bmatrix}
	\begin{bmatrix} 
		B \\
		ln(A)
	\end{bmatrix}
	=
	\begin{bmatrix} 
		ln(R(T_i))
	\end{bmatrix}
\]

\subsubsection{Polinomial}
\[
	V(T) = A + B * T + C * T^2 + D * T^3
\]

\[
\begin{bmatrix} 
	1 & T_i & T_{i}^2 & T_i^3 
\end{bmatrix}
\begin{bmatrix} 
	A \\ 
	B \\
	C \\
	D
\end{bmatrix}
= 
\begin{bmatrix} 
	V(T_{i})
\end{bmatrix}
\]



\begin{figure}[h!]
	\centering
	\includegraphics[width=0.8\columnwidth]{../CodigoPython/imagenes/PT1000_con_ajuste.png}
	\caption{Curva rango completo PT1000}
	\label{fig:1}
\end{figure}

\begin{figure}[h!]
	\centering
	\includegraphics[width=0.8\columnwidth]{../CodigoPython/imagenes/TYPE_E_con_ajuste.png}
	\caption{Curva rango completo TYPE E}
	\label{fig:2}
\end{figure}

\begin{figure}[h!]
	\centering
	\includegraphics[width=0.8\columnwidth]{../CodigoPython/imagenes/TMP235_con_ajuste.png}
	\caption{Curva rango completo TYPE TMP}
	\label{fig:3}
\end{figure}

\begin{figure}[h!]
	\centering
	\includegraphics[width=0.8\columnwidth]{../CodigoPython/imagenes/NTCLE100E3_con_ajuste.png}
	\caption{Curva rango completo NTCLE100E3338}
	\label{fig:4}
\end{figure}




\subsection{Determinación de curva caracteristica}

Para la determinación del rango de operación, y que este dentro de por lo menos el 60\% de todos los sensores, es necesario graficar todos los rangos de los sensores y traslaparlos, con el fin de encontrar un rango que todos posean dentro de sus datasheets, y que tambien sean de maximo el 60\%, a continuación la grafica y el rango escogido para la curva de temperatura del horno a experimentar con el elementos "X".\\

\begin{figure}[h!]
	\centering
	\includegraphics[width=0.8\columnwidth]{../CodigoPython/imagenes/rangos_sensores.png}
	\caption{Rangos de temperatura por sensor}
	\label{fig:5}
\end{figure}

El rango definido (rango deseado en la grafica \ref{fig:5}) para la simulación del horno y para tener una igualdad de rangos en el margen de los sensores según los rangos dados es de 0° a 100° Centigrados. Lo cual en una aplicación practica puede estar relacionado con un rango de cocina de un alimento, como por ejemplo la pasteurización de la leche (donde la leche se lleva a una temperatura que oscila entre los 55 y los 75 ºC durante 17 segundos).\\

Es posible analizar las magmnitudes de medición del rango de cada sensor, y que el rango escogido bajo las especificaciones del ejercicio, subdimensiona o le da bajo rendimiento a los sensores de mayor rango.


\subsection{Ecuación caracter\'istica para rango espec\'ifico}

En este apartado, es necesario aplicar un ajuste fino a cada uno de nuestros sensores, esto debido a que ya tenemos dos datos de gran importancia, el primero saber la curva de comportamiento de cada sensor y segundo el rango en el que vamos a trabajar, esta es una practica de uso general en aplicaciones de medición en todas las escalas, debido a que el fabricante propone una tabla de uso general y en el mayor rango posible, sin embargo, las aplicaciones dadas por los diferentes usuarios necesitan un ajuste o una curva de gran presicion en el proceso especifico.\\


Los parametros obtenido para cada uno de los sensores en el rango fueron:


\begin{table}[h]
	\centering
	\caption{Párametros de rango ajustado de cada sensor}
	\label{tab:my-table1}
	\resizebox{\columnwidth}{!}{%
		\begin{tabular}{|c|c|c|c|c|c|}
			\hline
			\textbf{Sensor}              & \textbf{Curva} & \textbf{A}         & \textbf{B}         & \textbf{C}         & \textbf{D}          \\ \hline
			PT1000                       & Lineal         & 3.85               & 1000.92            & -                  & -                   \\ \hline
			\multicolumn{1}{|c|}{TYPE E} & Polinomial     & $4.89\cdot e^{-5}$ & $5.86\cdot e^{-2}$ & $4.74\cdot e^{-5}$ & $-1.85\cdot e^{-8}$ \\ \hline
			TYPE TMP                     & Lineal         & 10                 & 500                & -                  & -                   \\ \hline
			NTCLE100E3338                & Exponencial    & 2873.3             & -8.45              & -                  & -                   \\ \hline
		\end{tabular}%
	}
\end{table}

% Please add the following required packages to your document preamble:
% \usepackage{graphicx}
\begin{table}[h]
	\centering
	\caption{Error RMSE (root mean square error) por sensor}
	\label{tab:my-table2}
	\resizebox{0.5\columnwidth}{!}{%
		\begin{tabular}{|c|c|}
			\hline
			\textbf{Sensor} & \textbf{Error RMSE} \\ \hline
			PT1000          & 0.4707             \\ \hline
			TYPE E          & 0.0002             \\ \hline
			TMP235-Q1        & 1.6126             \\ \hline
			NTCLE100E3338   & 0.3561             \\ \hline
		\end{tabular}%
	}
\end{table}

\subsection{Propagaci\'on de incertidumbre}

Dado que algunos de los fabricantes presentan el error en t\'erminos de temperatura, es necesario conocer el impacto en la medida del sensor, ya sea en t\'erminos de voltaje o resistencia. Por ende, se utilizar\'a la fo\'rmula de propagac\'on de incertidumbre de primero orden de una variable.

\begin{equation}
	\sigma_f^2 = \left( \frac{df}{dT} \right)^2 \sigma_T^2
\end{equation}

donde:
\begin{itemize}
	\item $\sigma_f^2$ es la varianza de la funci\'on $f$,
	\item $\frac{d f}{d T}$ es la derivada de la funci\'on respecto a la variable $T$,
	\item $\sigma_T^2$ es la varianza de la variable $T$.
\end{itemize}



\subsubsection{Funci\'on lineal}
	\[
		f(T) = A * T + B
	\]
	\[
		\frac{d f}{d T} = A
	\]
	\[
		\sigma_f^2 = A^2 \sigma_T^2
	\]

\subsubsection{Funci\'on exponencial}
	\[
	f(T) = A \cdot e^{\left( \frac{B}{T} \right)}
	\]
	\[
	\frac{d f}{d T} = A \cdot e^{\left( \frac{B}{T} \right)} \cdot \left( -\frac{B}{T^2} \right)
	\]
	\[
	\frac{d f}{d T} = -\frac{A B}{T^2} \cdot e^{\left( \frac{B}{T} \right)}
	\]
	\[
	\sigma_f^2 = \left( \frac{d f}{d T} \right)^2 \sigma_T^2
	\]
	\[
	\sigma_f^2 = \left( -\frac{A B}{T^2} e^{\left( \frac{B}{T} \right)} \right)^2 \sigma_T^2
	\]
	\[
	\sigma_f^2 = A^2 B^2 \frac{e^{\left( \frac{2B}{T} \right)}}{T^4} \sigma_T^2
	\]

\subsubsection{Funci\'on polinomial}
	
	\[
	f(T) = A + B \cdot T + C \cdot T^2 + D \cdot T^3
	\]
	Calculamos la derivada de \( F(T) \) respecto a \( T \):
	\[
	\frac{d f}{d T} = B + 2 C T + 3 D T^2
	\]
	Luego, aplicando la fórmula de propagación de incertidumbre:
	\[
	\sigma_f^2 = \left( \frac{d f}{d T} \right)^2 \sigma_T^2
	\]
	por lo tanto:
	\[
	\sigma_f^2 = \left( B + 2 C T + 3 D T^2 \right)^2 \sigma_T^2
	\]

Cada una de las incertidumbres anteriores ser\'an sumadas a sus correspondientes curvas caracter\'isticas. Adem\'as, se debe destacar que los errores, para el caso polinomial y exponencial, no es constante, y depender\'a de la temperatura medida en cada instante de tiempo.

\subsection{Errores}
	Cada fabricante nos presenta el error propio de cada sensor, este depende del tipo de medición, materiales tecnologia de construcción y demás, y luego probados en laboratorio, alli mediante un sensor de referencia de la más alta calidad y precisión es posible obtener los errores homocedasticos y heterocedasticos de cada sensor de la siguiente tabla.
	
	\begin{table}[h]
		\centering
		\caption{Errores proporcionados por los fabricantes para cada sensor}
		\label{tab:errores_sensores}
		\begin{tabular}{|c|c|}
			\hline
			Sensor & Error \\ \hline
			PT1000 & $\pm 0.5\ ^{\circ} \mathrm{C}$ \\ \hline
			TMP235-Q1 & $\pm2.5 ^{\circ}\mathrm{C}$ \\ \hline
			Tipo E & 
			\begin{tabular}[c]{@{}c@{}}
				Si $T > 0\ ^{\circ}\mathrm{C}$: $\pm 1.7\ ^{\circ}\mathrm{C}$ o $\pm 0.5\%$, el que sea mayor. \\ 
				Si $T < 0\ ^{\circ}\mathrm{C}$: $\pm 1.7\ ^{\circ}\mathrm{C}$ o $\pm 1.0\%$, el que sea mayor.
			\end{tabular} \\ \hline
			NTCLE100E3 & $ \pm 2\%$ sobre la medici\'on. \\ \hline
			
		\end{tabular} 
			
		
	\end{table}


\subsection{Error de medida}
	En nuestro caso, las medidas de cada sensor se ven afectadas por dos factores, tanto el error del ajuste como el error del fabricante. Dicho lo anterior, para encontrar el error se realizara una suma de errores.
	
	\[
		\Delta f = \pm \sqrt{\Delta f_{fabricante} + \Delta f_{ajuste}}
	\]
	
	
	
	
	
\subsection{Simulación de horno}

En este punto es necesario entonces que mediante el rango escogido para los sensores se realice una curva caracteristica del proceso, en este caso la curva es:


\begin{figure}[h!]
	\centering
	\includegraphics[width=0.8\columnwidth]{../CodigoPython/imagenes/temperatura_horno.png}
	\caption{Perfil de temperatura del horno.}
	\label{fig:5}
\end{figure}

Es en esta curva donde se probara el comportamiento y la salida de cada uno de los sensores, se decide hacer una función compuesta por varios sub tramos, en las cuales en su mayoria son de orden 2 y demas, con el objetivo de no ver un comportamiento tradicional de curvas lineales de aumento y disminución de temperatura.

\subsection{Operación de los sensores}

En este apartado, para la operación de los sensores se asumira un comportamiento Gaussiano para todos, y en el siguiente apartado un comportamiento del error diferente para cada uno de los sensores, en este proceso es posible la inclusión de outliers con el 0.5\% en cada subpunto.

\subsubsection{Comportamiento Gaussiano}

El comportamiento gaussiano se simulo utilizando los errores RMSE del ajuste del cada sensor en el rango de trabajo previamente descrito, y las componentes homocedastiscas y heterocedastica de cada sensor, incluidas dentro de una randomizador normalizado, de media cero y desviación segúm el error, cabe resaltar que mientras los errores RMSE y homocedastica son solo valores sumados a la función, la comoponente heterocedastica se debe multiplicar ya que se suele expresar como un porcentaje de la medida actual del sensor, asi mismo a mayor valor de la magnitud de medida de mayor tamaño es el error.

\begin{figure}[h!]
	\centering
	\includegraphics[width=0.8\columnwidth]{../CodigoPython/imagenes/PT1000_simulado.png}
	\caption{Comportamiento PT1000.}
	\label{fig:PT1000_simulado}
\end{figure}

\begin{figure}[h!]
	\centering
	\includegraphics[width=0.8\columnwidth]{../CodigoPython/imagenes/TMP235_simulado.png}
	\caption{Comportamiento TMP235.}
	\label{fig:TMP235_simulado}
\end{figure}

\begin{figure}[h!]
	\centering
	\includegraphics[width=0.8\columnwidth]{../CodigoPython/imagenes/TYPE_E_simulado.png}
	\caption{Comportamiento TYPEE.}
	\label{fig:TYPE_E_simulado}
\end{figure}

\begin{figure}[h!]
	\centering
	\includegraphics[width=0.8\columnwidth]{../CodigoPython/imagenes/NTCLE100E3_simulado.png}
	\caption{Comportamiento NTCLE100E3.}
	\label{fig:NTCLE100E3_simulado}
\end{figure}

Es posible analizar de las graficas unos comportamientos poco pensados si veiamos la tabla \ref{tab:errores_sensores} sobre el error de cada sensor en el rango de operación, viendo que por ejemplo el PT1000 apesar de tener un peor desempeño en el ajuste de curva, en un evento simulado se siñe bien a la forma de curva del horno, al giual que el sensor NTCLE100E3, sin embargo, el sensor type E, no tuvo tan buen rendimiento como se podia asumir viendo la tabla en referencia.



\textbf{Comportamiento del error}


\begin{figure}[h!]
	\centering
	\includegraphics[width=0.8\columnwidth]{../CodigoPython/imagenes/error_PT1000.png}
	\caption{Comportamiento error PT1000.}
	\label{fig:error_PT1000}
\end{figure}

\begin{figure}[h!]
	\centering
	\includegraphics[width=0.8\columnwidth]{../CodigoPython/imagenes/error_TMP235.png}
	\caption{Comportamiento error TMP235.}
	\label{fig:error_TMP235}
\end{figure}

\begin{figure}[h!]
	\centering
	\includegraphics[width=0.8\columnwidth]{../CodigoPython/imagenes/error_TYPE_E.png}
	\caption{Comportamiento error TYPEE.}
	\label{fig:error_TYPE_E}
\end{figure}

\begin{figure}[h!]
	\centering
	\includegraphics[width=0.8\columnwidth]{../CodigoPython/imagenes/error_NTCLE100E3.png}
	\caption{Comportamiento error NTCLE100E3.}
	\label{fig:error_NTCLE100E3}
\end{figure}

La explicación al parrafo anterior y el comportamiento de los sensores, se puede deber a sus componentes intrinsecas de fabricación, es decir, el error homocedastico y heterocedastico, el error de ajuste puede hasta quedar relegado en grandes magnitudes del sensor en comparación con estas otras magnitudes, lo interesante de la curva del horno, es que permite evaluar el comportamiento de los sensores en dinamismo, y en cambios "subitos" de temperatura, con ajustes y variaciones constantes.


\subsubsection{Comportamiento Alterno}

En este apartado, se evalua el comportamiento de cada sensor, esta vez integrando las componentes de error, no solo de manera lineal, sino en distribución tipo Poisson, Uniforme y Laplace.

\begin{figure}[h!]
	\centering
	\includegraphics[width=0.8\columnwidth]{../CodigoPython/imagenes/PT1000_ruido_gaussiano.png}
	\caption{Comportamiento PT1000 con ruido gaussiano.}
	\label{fig:PT1000_ruido_gaussiano}
\end{figure}

\begin{figure}[h!]
	\centering
	\includegraphics[width=0.8\columnwidth]{../CodigoPython/imagenes/TMP235_ruido_poisson.png}
	\caption{Comportamiento TMP235 ruido poisson.}
	\label{fig:TMP235_ruido_poisson}
\end{figure}

\begin{figure}[h!]
	\centering
	\includegraphics[width=0.8\columnwidth]{../CodigoPython/imagenes/TYPE_E_ruido_uniforme.png}
	\caption{Comportamiento TYPEE con ruido uniforme.}
	\label{fig:TYPE_E_ruido_uniforme}
\end{figure}

\begin{figure}[h!]
	\centering
	\includegraphics[width=0.8\columnwidth]{../CodigoPython/imagenes/NTCLE100E3_ruido_laplace.png}
	\caption{Comportamiento NTCLE100E3 ruido laplace.}
	\label{fig:NTCLE100E3_ruido_laplace}
\end{figure}

Es curioso en este caso ver el comportamiento de cada uno de los sensores, y como "mejoraron" su rendimiento en comparación con todas las componentes del error gaussianas, esto debido a como distribuye las magnitudes y componentes erroneas usadas, cabe resaltar que curiosamente se ve más influencia de los outliers en este apartado, aunque en ambas tengan la misma probabilidad.

\section{Conclusiones}

\begin{itemize}
	\item Comprender la metodología del ajuste fino de sensores resulta fundamental para su aplicación en entornos complejos y diversos como la industria, la atmósfera o los laboratorios. Esta técnica permite personalizar los sistemas de medición según las necesidades específicas de cada contexto. Además, el ajuste de rangos mejora significativamente el rendimiento de sensores calibrados en fábrica, ya que optimizar sus parámetros dentro de un subrango puede aumentar la precisión de la medición. Este beneficio suele lograrse a cambio de una leve penalización en el error de ajuste, que en muchos casos resulta despreciable frente a la ganancia en certeza.
\end{itemize}

\begin{thebibliography}{00}

\item Clases magistrales e información suministrada en la clase de Sistemas de medición avanzada, Maestria en ingenieria electrica, semestre 2025-1

\end{thebibliography}

\end{document}
